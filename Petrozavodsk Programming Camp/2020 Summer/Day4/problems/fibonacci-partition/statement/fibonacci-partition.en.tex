\begin{problem}{Fibonacci Partition}{\textsl{standard input}}{\textsl{standard output}}{5 seconds}{256 mebibytes}

The sequence of Fibonacci numbers is defined as:
$$
F_n=\begin{cases}
1 & n=1 \\
2 & n=2 \\
F_{n-1}+F_{n-2} & \text{otherwise}
\end{cases}
$$

The first few elements of the sequence are $1, 2, 3, 5, 8, 13, 21, 34, \dots$

For a given positive integer $n$, let $\mathit{partition}(n)$ be the maximum value of $m$ such that $n$ can be expressed as a sum of $m$ distinct Fibonacci numbers.
For example, $\mathit{partition}(1) = \mathit{partition}(2) = 1$, $\mathit{partition}(3) = \mathit{partition}(4) = \mathit{partition}(5) = \mathit{partition}(7) = 2$, $\mathit{partition}(6) = \mathit{partition}(8) = 3$.

Chiaki has an integer $X$ which initially equals to $0$. She will perform some operations on $X$: the $i$-th operation will add $a_i \cdot F_{b_i}$ to $X$. 

After each operation, Chiaki would like to know the value of $\mathit{partition}(X)$. It is guaranteed that, after each operation, $X$ will be a positive integer.

\InputFile
There are multiple test cases. The first line of input contains an integer $T$, indicating the number of test cases. For each test case:

The first line contains an integer $n$ ($1 \le n \le 5 \cdot 10^4$): the number of operations.

Each of the next $n$ lines contains two integers $a_i$ and $b_i$ ($1 \le |a_i|, b_i \le 10^9$).

It is guaranteed that the sum of $n$ for all test cases will not exceed $5 \cdot 10^4$.


\OutputFile
For each test case, output $n$ integers: the $i$-th integer denotes the value of $\mathit{partition}(X)$ after the $i$-th operation.




\Example

\begin{example}
\exmpfile{example.01}{example.01.a}%
\end{example}

\Note
The value of $X$ after each operation in the sample: $1, 2, 4, 7, 12, 20, 33, 54, 88, 72$.

\end{problem}
