\begin{tutorial}{Ternary String Counting}

Let's try the $O(n^3)$ solution first. Let $\mathit{ways}(i,j,k)$ be the number of ways that we have to fill the values of $s_1,s_2,\ldots,s_i$, and the nearest two different characters are at position $j$ and $k$ ($i > j > k$).

Each of the $m$ restrictions actually determines the ranges of $j$ and $k$. For each valid $j$ and $k$ that: $j \in [\mathit{jmin}_i, \mathit{jmax}_i]$ and $k \in [\mathit{kmin}_i, \mathit{kmax}_i]$, we have the following transitions:

\begin{enumerate}
\item $\mathit{ways}(i+1, i, k) \stackrel{+}{\leftarrow} \mathit{ways}(i, j, k)$
\item $\mathit{ways}(i+1, i, j) \stackrel{+}{\leftarrow} \mathit{ways}(i, j, k)$
\item $\mathit{ways}(i+1, j, k) \stackrel{+}{\leftarrow} \mathit{ways}(i, j, k)$
\end{enumerate}

We can see easily that the first dimension of $\mathit{ways}(\cdot,\cdot,\cdot)$ is useless: in the first two cases, the second dimension will always be $i$, and in the third case, $j$ and $k$ will never change.

Actually the transitions are equivalent to: for a given rectangle $R$, make all the entries outside $R$ be zero and:

\begin{enumerate}
\item find the sum of a specific row,
\item find the sum of a specific column,
\item do not change the values.
\end{enumerate}

We can also see that, if an entry becomes zero, it will always be zero. And for a fixed row, the non-zero entries form a consecutive range.

For each row $i$ of the dynamic programming array, we can maintain the boundaries $\mathit{left}_i$ and $\mathit{right}_i$ of the non-zero entries. For each row and column, we can maintain the row-sum and column-sum.

For each rectangle $R$, we can just clear the entries for each row and maintain the row-sum or column-sum in the mean time.

The time and space complexity is $O(n^2)$.


\end{tutorial}
