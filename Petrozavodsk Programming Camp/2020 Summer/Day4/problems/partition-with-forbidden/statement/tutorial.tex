\begin{tutorial}{Partition Number}

Let $p(m)$ be the number of solutions of equation $x_1+x_2+\ldots+x_k=m$ such that $x_1 \le x_2 \le \ldots \le x_k$. Actually, it is the partition number. And let $\mathit{odd}(x)$ and $\mathit{even}(x)$ be the number of ways summing up to $x$ using odd or even distinct numbers from $A$, respectively.

The answer for the problem will be:

$$
\sum_{i=0}^{m} (\mathit{even}(i) - \mathit{odd}(i)) \cdot p(m-i)
$$

which can be proved using inclusion-exclusion principle.

The value of $\mathit{odd}(x)$ and $\mathit{even}(x)$ can be calculated using simple dynamic programming. And the value of $p(x)$ can be calculated using the classic dynamic programming or pentagonal number theorem in $O(m \sqrt m)$. Also, polynomial inversion and fast Fourier transform work in $O(m \log m)$.

\end{tutorial}
