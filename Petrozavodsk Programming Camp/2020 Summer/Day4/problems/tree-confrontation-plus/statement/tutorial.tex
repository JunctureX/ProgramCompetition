\begin{tutorial}{Tokens on the Tree}

Let's try to fix $w$ and $b$ first. And without loss of generality, we can assume $w \ge b$.

Let $S$ be the set of vertices which must be in the white connected component.

\begin{itemize}
\item $S \ne \varnothing$. After deleting all vertices in $S$, we will have some connected components. We can see that $f(w,b)$ will be the number of connected components whose size is at least $b$.
\item $S = \varnothing$. We can see that $f(w,b)$ will only be $1$ or $2$. And $f(w,b)$ will be $1$ if and only if there exists a vertex $u$ such that, after deleting this vertex, there exist at least three connected components, the size of two of them $\ge w$ and the size of the remaining one $\ge b$.
\end{itemize}

We can solve the problem based on the above observation.

Let's try to fix the value of $w$. A vertex $x$ must be in the white connected component if an only if the maximum size of the subtree of $x$ is less than or equal to $w$.

When $w=n$, all the vertices are in the white connected component. As $w$ decreases, the size of $S$ will decrease too. We can use a disjoint-set data structure to maintain the sizes of the connected components.

For the fixed $w$, if $S$ is nonempty, we need to find the value of 
$$
w \cdot \left(\sum\limits_{b=1}^{\min(w,n-w)} b \cdot f(w, b)\right)\text{.}
$$

When maintaining the size of the remaining connected components, we can use a segment tree or binary index tree to maintain the value of $s(x)$, which means the number of connected components whose size $\ge x$, and the value of $s(x) \cdot x$.

The value of $\sum\limits_{b=1}^{\min(w,n-w)} b \cdot f(w, b)$ is just a range sum in the segment tree or binary index tree.

If $S$ is empty, we need to maintain some other information. We only need to know the number of $b$ which makes $f(w,b)=1$. For each vertex $x$, we need to maintain whether it has at least two subtrees with size $\ge w$ and the size of the third largest subtrees.

For each vertex $x$, we maintain the sorted list $L_x$ of subtree in the decreasing order of subtree size. As $w$ decreases, we can pop the corresponding subtree from each $L_x$ and find the maximum value $\mathit{bound}$ of the size of the third largest subtree.

For $b \le \min(w, \mathit{bound})$, the value of $f(w,b)$ is $1$. And for $\mathit{bound} < b \le w$, the value of $f(w,b)$ is $2$.

The time complexity is $O(n \log n)$.

\end{tutorial}
