\begin{tutorial}{Necklace}

Let the number of gems of color $i$ in the the necklace be $\mathit{cnt}(i)$. A necklace of length $m$ is good iff $m \ge 2 \cdot \max_i(\mathit{cnt}(i))$.

Let the maximum value of $\mathit{cnt}(i)$ be $x$. For gems with color $i$, only the largest $x$ values could be on the necklace: let it be $V_i$. 

Let $V$ be $V_1 \cup V_2 \cup \ldots \cup V_n$. If we sort the values in $V$ in decreasing order, the first $2x$ (except $x=1$, we need at least $3$ values) must be taken. For the remaining values, we can take all positive values. This will definitely give us the best solution.

We can use priority queue to speed up the procedure. When $x$ becomes $x+1$, only new values will be added to $V$. We can use a minimum heap $A$ to maintain the first $2x$ values, and a maximum heap $B$ to maintain the remaining values. For the newly added values, we add them to $B$ first. After that, take top values from $B$ to make the size of $A$ at least $2(x+1)$. Also, take top positive values from $B$ to $A$.

As for the solution, we can find the optimal $x$ and try to simulate the above procedure using the optimal $x$ to get a valid solution.

\end{tutorial}
